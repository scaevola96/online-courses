\input{preamble}
\begin{document}

\title{Определение ширины запрещенной зоны полупроводника}
\thanks{11.1}

\author{Барсегян Сергей
}
\affiliation{
 Московский Физико-Технический Институт\\
 Физтех-школа физики и исследований им. Ландау, 685\\
}
%\date{\today}

\begin{abstract}
Исследуется температурная зависимость проводимости типичного полупроводника - германия или кремния. Определяется ширина запрещенной зоны с помощью универсального цифрового вольтметра.

\end{abstract}

\pacs{Valid PACS appear here}

\maketitle

Измерим проводимость полупроводникового и медного образцов в зависимости от температуры.

Будем нагревать образцы от комнатной температуры до $100^{\circ}C$. Через каждые $5^{\circ}C$ будем измерять сопротивление полупроводникового и медного образцов, поочередно подключая их к прибору с помощью ключа K.

Используя соотношение
$$ \sigma = \frac{l}{RS},$$

построим график зависимости $\sigma(T)$ для обоих образцов. По наклону графика для медного образца определим температурный коэициент сопротивления меди.
\includegraphics[scale=0.20]{my_plot1.png}
\begin{gather*}
\alpha_{cu} = \frac{1}{R}\frac{dR}{dT} = \sigma\frac{d}{dT}(\frac{1}{\sigma}) = - \frac{1}{\sigma}\frac{d\sigma}{dT} =  (3.5 \pm 0.2)~10^{-3}\cdot 1/K.
\end{gather*}
Построим график $\ln(\sigma) = f(1/T)$ для полупроводникового образца и по наклону его прямолинейной части (при более высоких температурах) определим ширину запрещенной зоны; выразим ее в электрон-вольтах.
\includegraphics[scale=0.20]{my_plot2.png}
\begin{gather*}
\tg(\alpha) = \frac{\Delta}{2k_B} = 4.2 \cdot 10^{3}~K, \\
\Delta = (0.7 \pm 0.1)~eV.
\end{gather*}

По полученным значениям проводимости и ширины запрещенной зоны определим материал полупроводникового образца. В нашем случае можно сделать вывод о том, что в экспериментальной установке использовался германий.

Оценим достоверность полученных результатов в определении ширины запрещенной зоны исследуемого полупроводника и коэиффциента температурного сопротивления меди.

\begin{gather*}
\alpha_{cu} = 3.8\cdot10^{-3}~1/K, \\
\Delta_{table}^{Ge} = (0.67 \pm 0.1)~eV.
\end{gather*}

\includegraphics[scale=0.20]{my_plot3.png}

\end{document}